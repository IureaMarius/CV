\documentclass[a4paper,hidelinks,11pt]{article}


\usepackage{titlesec}
\usepackage{titling}
\usepackage[margin=1in]{geometry}
\usepackage{parskip}
\usepackage[hidelinks]{hyperref}
\usepackage{helvet}
\pagenumbering{gobble}
\titleformat{\section}
{\huge\bfseries}
{\hspace{-0.6cm}}
{0em}
{}[\titlerule]

\titleformat{\subsection}[runin]
{\bfseries\large}
{$\bullet$}
{0.4em}
{}[ ---]

\titlespacing{\section}
{0em}{1em}{1em}


\titlespacing{\subsection}
{-1em}{0.8em}{0.5em}

\titlespacing{\subsubsection}
{0em}{0.6em}{0.1em}

\titleformat{\subsubsection}
{\bfseries}
{}
{0em}
{}

\begin{document}
\title{Curriculum Vitae}
\author{Iurea Marius}
\renewcommand{\maketitle} 
{
        \begin{center}
        {\huge\bfseries
        \thetitle

        \theauthor}

        marius.iurea@yahoo.com --- 0741055112 --- \href{https://www.linkedin.com/in/marius-iurea-49b7021b5/}{LinkedIn}

        \end{center}
}

\maketitle

\section{Personal Profile}
Dedicated, self-motivated, and passionate software engineer, looking to fill a position where I can grow and take on new
challenges. I am a fast learner and able to adapt to different work environments. I am confident that if given the opportunity, I can be a useful part of your software engineering team. \href{https://github.com/IureaMarius/CV}{(Up-to-date CV)}



\section{Technical Skills}
\subsection{Languages}
C\#, JavaScript, C, C++, Python
\subsection{Frameworks/Libraries}
ASP.NET, Node.js, Express.js, JQuery, Django
\subsection{Tools}
Git, Unity, Visual Studio, Vim, Nuget, NPM

\section{Experience}
\subsection{IOmundo} 01/10/2020 $\rightarrow$ current
\subsection{Job Title:} Junior Software Engineer
\subsubsection{Project Overview:}
Mostly worked on developing an integrated booking engine using Angular, to be embeded in our client's websites. Worked on fixing issues, adding features, and improving our web services
\subsubsection{Tech Stack:} C\#, Angular
\subsubsection{Responsibilities:}
-learned Angular over the course of 2 weeks while discussing the project requirements

-created a bridge to convert our existing XML based API into a JSON based API

-added endpoints to expose our backend Umbraco CMS

-changed the caching to include a new type of 0 availability services

-fixed various bugs in our backend

-added internationalization support

-optimized the front end rendering of search results by adding virtual scrolling

-found an issue in our content retrieval Umbraco controller that caused a big part of the content stored in Umbraco to be retrieved from the database, instead of the more efficient XML cache. After adding an Lucene search index, and making sure all of the calls tried to hit the cache before going to the database, reduced the load time of an average search that retrieved 200 services from about 90 seconds to less than 8 seconds.

-integrated the DIBSpayment Easy API to our booking engine, in the angular application and in the backend ERP product

-added automatic image resizing and watermarking of uploaded images to our Umbraco CMS

-helped create and run various exports from our client's old system to our CMS

-fixed a bug that caused a change in hue of our resized images

-had various calls with our client, talking about requirements, managing expectations, and deciding on the final UX.

-in another project, worked on a web scraper that kept the data from our client's platform in sync with ours


\subsection{Bitdefender} 01/06/2020 $\rightarrow$ 01/10/2020
\subsection{Job Title:} Junior Software Engineer
\subsection{Department:} Cyber Threat Intelligence Lab
\subsubsection{Project Overview:}
The application is an internally used tool that deals with the analysis of logs created by other Bitdefender products: it filters out irrelevant events and gives the user tools to analyze the chain of events that led up to a malware detection event.
\subsubsection{Technologies used: Node.js, Express.js, JQuery, Bootstrap, AJAX, JSON}
\subsubsection{Responsibilities:}
-went through a 2-week learning program where I learned how to use Node.js and Express.js where I was familiarized the most commonly used Node packages

-refactored an existing codebase for a log processing and analysis utility

-modularized the application and made it comply with the code style guide

-rebuilt the complex filtering and preprocessing of the JSON files.

-substantially optimizing the filtering and lowered general load times across the application by rewriting the glob filtering modules and modules modeling the relations between events and detections

-took initiative in implementing lazy loading in order to decrease the size of the requests sent and received, drastically decreasing server and client load

-added support for a different type of log file

-implemented automatic detection of log file type and OS



\section{Education}
\subsection{Currently getting a bachelor's degree at 
the Faculty of Computer Science Iasi(final year student)}
 8.5/10 Grade Average


\subsection{Winning project in FIIPractic full-stack .Net courses}

The technologies we used were ASP.NET, Microsoft SQL Server, HTML, CSS, Bootstrap, Javascript,
JQuery, AJAX, and Git. 

\end{document}




